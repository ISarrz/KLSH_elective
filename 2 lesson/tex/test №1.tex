%! Author = zero
%! Date = 29/07/2024

% Preamble
\documentclass[a4paper, 12pt]{article}

\usepackage[english,russian]{babel}
\usepackage[T2A]{fontenc}
\usepackage[utf8]{inputenc}
\usepackage{geometry}
\usepackage{enumitem}
\usepackage{setspace}
\usepackage{amssymb}
\usepackage{graphicx}
\usepackage{float}
\usepackage{wrapfig}
\geometry{top=5mm}
\renewcommand{\arraystretch}{1.2}
\linespread{1}

% Document
\begin{document}
    \begin{center}
        \textbf{
            Самостоятельная работа №1\\
            Основы алгебры логики}
    \end{center}

    \begin{enumerate}
        \item Найти значения: $C, D, E, F$, если $A = 0, B = 1$\\
        \begin{enumerate}
            \item $C = (A \vee B) \wedge A \vee B \wedge \overline A$\\

            \item $D = (C \vee A \vee \overline B) \wedge (C \wedge \overline B \vee A) \vee B \vee \overline C$\\

            \item $E$ = Я сегодня позавтракал \textbf{или} (Сегодня ночью я гулял \textbf{и} Сегодня ночью я не гулял) \textbf{или} Я в КЛШ.\\

            \item $F = E \wedge A \vee (1 \vee 0 \vee 1) \wedge \overline B \vee 0$\\

        \end{enumerate}

        \item Построить таблицу истинности:
        \begin{enumerate}
            \item $F = A \vee B \wedge \overline B$

            \item $F = (B \vee C) \wedge (\overline A \wedge C \vee \overline C)$
        \end{enumerate}

        \item Решить кругами Эйлера:
        \begin{enumerate}
            \item $A \wedge B \vee C$

            \item $Z \vee A \wedge C \vee \overline B$
        \end{enumerate}

        \item Найти $X, Y$ при которых выражение $X \wedge Y \vee Y = 1$
    \end{enumerate}

\end{document}