%! Author = zero
%! Date = 29/07/2024

\documentclass[a4paper, 12pt]{article}

\usepackage[english,russian]{babel}
\usepackage[T2A]{fontenc}
\usepackage[utf8]{inputenc}
\usepackage{geometry}
\usepackage{enumitem}
\usepackage{setspace}
\usepackage{amssymb}
\usepackage{graphicx}
\usepackage{float}
\usepackage{wrapfig}
\geometry{top=5mm}
\renewcommand{\arraystretch}{1.2}
\linespread{1}

% Document
\begin{document}
    \begin{center}
        \textbf{Задачи №1 Основы алгебры логики}
    \end{center}

    \begin{center}
        \textbf{№1 Переписать в символьном виде и посчитать}
    \end{center}

    \begin{enumerate}
        \item Я в КЛШ \textbf{и} Дождь \textbf{не} идет только днем.

        \item Ваня Адо вожатый команды йота \textbf{или} Cолнце светит днем.

        \item Трава синяя \textbf{или} Деревья \textbf{не} растут вниз.

        \item Сегодня был дождь \textbf{и} Сегодня \textbf{не} был дождь.

        \item Сегодня вечерний клуб \textbf{или} Я \textbf{не} иду на турнички \textbf{или}
        Сегодня \textbf{не} вечерний клуб \textbf{и} Я иду на турнички.

        \item Зондеры голодают \textbf{и} Зондеры не спят \textbf{или} Солнце светит днем \textbf{и} {Зондер пашет}.

        \item Я отдаю свой полдник Зондеру \textbf{или}
        Я съедаю полдник \textbf{и} (иду спать
        \textbf{или} иду решать задачи дня \textbf{или} у меня свободное время \textbf{или} небо голубое)

        \item  $5 > 6 \textbf{ и } 7 < 2$

    \end{enumerate}
    \begin{center}
        \textbf{№2 Решить кругами Эйлера}
    \end{center}

    \begin{enumerate}
        \item $A \wedge B$
        \item $\overline A \vee B$
        \item $A \wedge B \wedge C$
        \item $A \wedge \overline B$
    \end{enumerate}

    \begin{center}
        \textbf{№3 Построить таблицу истинности}
    \end{center}

    \begin{enumerate}
        \item $\overline A \wedge \overline B$
        \item $A \vee B \wedge \overline C$
        \item $A \wedge C \wedge B \wedge Z$
        \item $A \wedge \overline A \vee B \vee \overline B$

        \item $(A \vee \overline B) \wedge (\overline A \wedge B)$
        \item $(\overline A \vee B) \wedge (A \vee \overline B)$
    \end{enumerate}

    \begin{center}
        \textbf{№3 Решить уравнение}
    \end{center}

    \begin{enumerate}
        \item $(x \vee y) \wedge \overline x = 1$
        \item $(x \wedge y \vee x) \wedge(\overline z \wedge y \vee x) = 0$
        \item $\overline x \wedge y \vee x \vee y = 1$
        \item $x \wedge y \vee \overline y = 0$
    \end{enumerate}

\end{document}