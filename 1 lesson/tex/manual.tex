\usepackage{wasysym}
\usepackage{stmaryrd}%! Author = zero
%! Date = 09/07/2024

\documentclass[a4paper, 12pt]{article}

\usepackage[english,russian]{babel}
\usepackage[T2A]{fontenc}
\usepackage[utf8]{inputenc}
\usepackage{geometry}
\usepackage{enumitem}
\usepackage{setspace}
\usepackage{amssymb}
\usepackage{graphicx}
\usepackage{float}
\usepackage{wrapfig}
\geometry{top=5mm}
\renewcommand{\arraystretch}{1.2}
\linespread{1}

% Document
\begin{document}
    \renewcommand{\arraystretch}{1.5}

    \begin{minipage}[t]{0.2\textwidth}
        \textbf{Аксиомы}
        \centering
        \begin{enumerate}
            \item $\overline{\overline x} = x$
            \item $x \vee \overline x = 1$
            \item $x \vee 1 = 1$
            \item $x \vee x = x$
            \item $x \vee 0 = x$
            \item $x \wedge \overline x = 0$
            \item $x \wedge x = x$
            \item $x \wedge 0 = 0$
            \item $x \wedge 1 = x$
        \end{enumerate}

    \end{minipage}
    \begin{minipage}[t]{0.4 \textwidth}
        \centering
        \textbf{Операции}
        \begin{enumerate}
            \item Дизъюнкция $\vee$\\0 1 1 1

            \item Конъюнкция $\wedge$\\0 0 0 1

            \item Исключающее <<или>> $\oplus$\\ 0 1 1 0\\
            $x \oplus y = \overline x \wedge y \vee x \wedge \overline y$\\
            $x \oplus y = (\overlint x \vee \overline y) \wedge (x \vee y)$\\
            $x \oplus 1 = \overline x$\\
            $x \oplus 0 = x$\\
            $x \oplus x = 0$\\
            $(x \oplus y) \oplus y = x$\\
            $\overline x \oplus y = x \oplus \overline y$\\
            $x \oplus y = z; x \oplus z = y$

            \item Штрих Шеффера $|$\\ 1 1 1 0\\
            $x \mid y = \overline{x \wedge y} = \overline x \vee \overline y$\\
            $x \mid x = \overline x$\\

            \item Стрелка Пирса $\downarrow$\\ 1 0 0 0\\
            $x \downarrow y = \overline{x \vee y} = \overline x \wedge \overline y$\\
            $x \downarrow x = \overline x$

            \item Импликация $\rightarrow$\\ 1 1 0 1\\
            $x \rightarrow y = \overline x \vee y$\\
            $x \rightarrow 0 = \overline x$\\
            $x \rightarrow x = 1$\\

            \item Эквивалентность $\leftrightarrow$\\ 1 0 0 1\\
            $x \leftrightarrow y = (x \rightarrow y) \wedge (y \rightarrow x)$\\
            $x \leftrightarrow y = (x \wedge y) \vee (\overline x \wedge \overline y)$\\
            $x \leftrightarrow y = (\overline x \vee y) \wedge (x \vee \overline y)$\\
            $x \leftrightarrow x = 1$\\
            $x \leftrightarrow 0 = \overline x$\\
        \end{enumerate}

    \end{minipage}
    \begin{minipage}[t]{0.5\textwidth}
        \centering
        \textbf{Свойства логических операций}
        \begin{enumerate}
            \item Коммутативность\\
            $x \circ y = y \circ x,$\\
            $\circ \in \{\wedge, \vee, \oplus, \downarrow, \leftrightarrow, | \}$

            \item Идемпотентность\\
            $x \circ x = x$\\
            $\circ \in \{\vee, \wedge \}$

            \item Ассоциативность\\
            $(x \circ y) \circ z = x \circ (y \circ z),$\\
            $\circ \in \{\vee, \wedge, \oplus, \leftrightarrow \}$\\

            \item Дистрибутивность\\
            $x \wedge (y \vee z) = (x \wedge y) \vee (x \wedge z)$\\
            $x \vee (y \wedge z) = (x \vee y) \wedge (x \vee z)$\\
            $x \wedge (y \oplus z) = (x \wedge y) \oplus (x \wedge z)$

            \item Законы де Моргана\\
            $\overline{x \wedge y} = \overline x \vee \overline y = x \mid y$\\
            $\overline{x \vee y} = \overline x \wedge \overline y = x \downarrow y$

            \item Законы поглощения\\
            $x \wedge (x \vee y) = x$\\
            $x \vee (x \wedge y) = x$

        \end{enumerate}
    \end{minipage}

\end{document}