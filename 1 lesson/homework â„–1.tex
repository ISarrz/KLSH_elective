%! Author = zero
%! Date = 29/07/2024

\documentclass[a4paper, 12pt]{article}

\usepackage[english,russian]{babel}
\usepackage[T2A]{fontenc}
\usepackage[utf8]{inputenc}
\usepackage{geometry}
\usepackage{enumitem}
\usepackage{setspace}
\usepackage{amssymb}
\usepackage{graphicx}
\usepackage{float}
\usepackage{wrapfig}
\geometry{top=5mm}
\renewcommand{\arraystretch}{1.2}
\linespread{1}

% Document
\begin{document}
    \begin{center}
        \textbf{Домашняя работа №1 Основы алгебры логики}
    \end{center}

    \begin{enumerate}
        \item $A = 1, B = 0$\\
        Чему равно $A \wedge B \vee \overline B$

        \item $A = 1, B = 1$\\
        Чему равно $(A \vee \overline B) \wedge (\overline A \wedge B)$

        \item $A = 0, B = 0, C = 1$\\
        Чему равно $A \vee (\overline B \wedge C \wedge A) \wedge \overline A$

        \item $A = 1, B = 1, C = 1$\\
        Чему равно $\overline A \vee (\overline C \wedge \overline B) \wedge A \vee (A \wedge B \wedge C)$

        \item $A = $ Сегодня был дождь, $B = $ На завтрак оладьи\\
        Как выглядит символьная запись высказывания: <<Сегодня был дождь \textbf{и} На завтрак \textbf{не} оладьи>>\\
        (перевести в символьную запись)

        \item $A = $ Сегодня вечерний клуб, $B = $ Я иду на турнички\\
        Как выглядит символьная запись высказывания: <<Сегодня вечерний клуб \textbf{и} Я \textbf{не} иду на турнички \textbf{или} Сегодня \textbf{не} вечерний клуб \textbf{и} Я иду на турнички>>\\
        (перевести в символьную запись)

        \item Чему равно высказывание: <<Зондеры голодают \textbf{и} Зондеры не спят \textbf{или} Солнце светит днем \textbf{и} {Зондер пашет}>>\\
        (ответ 0 или 1)

        \item Чему равно высказывание: <<Я отдаю свой полдник Зондеру \textbf{или} Я съедаю полдник \textbf{и} (иду спать \textbf{или} иду решать задачи дня \textbf{или} у меня свободное время \textbf{или} небо голубое)>>\\
        (ответ 0 или 1)

        \item Найдите $X$ и $Y$ при которых выражение $(X \vee Y) \wedge \overline X$ истинно

        \item Найдите $X$, $Y$, $Z$ при которых выражение $(X \wedge Y \vee Z) \wedge(\overline Z \wedge Y \vee X)$ ложно
    \end{enumerate}


\end{document}