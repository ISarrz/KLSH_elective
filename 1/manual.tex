\usepackage{wasysym}%! Author = zero
%! Date = 09/07/2024

\documentclass[a4paper, 12pt]{article}

\usepackage[english,russian]{babel}
\usepackage[T2A]{fontenc}
\usepackage[utf8]{inputenc}
\usepackage{geometry}
\usepackage{enumitem}
\usepackage{setspace}
\usepackage{amssymb}
\usepackage{graphicx}
\usepackage{float}
\usepackage{wrapfig}
\geometry{top=5mm}
\renewcommand{\arraystretch}{1.2}
\linespread{1}

% Document
\begin{document}
    \renewcommand{\arraystretch}{1.5}

    \begin{minipage}[t]{0.2\textwidth}
        \textbf{Аксиомы}
        \centering
        \begin{enumerate}
            \item $\overline{\overline x} = x$
            \item $x \vee \overline x = 1$
            \item $x \vee 1 = 1$
            \item $x \vee x = x$
            \item $x \vee 0 = x$
            \item $x \wedge \overline x = 0$
            \item $x \wedge x = x$
            \item $x \wedge 0 = 0$
            \item $x \wedge 1 = x$
        \end{enumerate}

    \end{minipage}
    \begin{minipage}[t]{0.33\textwidth}
        \centering
        \textbf{Операции}
        \begin{enumerate}
            \item Дизъюнкция $\vee$\\0 1 1 1
            \item Конъюнкция $\wedge$\\0 0 0 1
            \item Эквивалентность $\leftrightarrow$\\ 1 0 0 1
            \item Импликация $\rightarrow$\\ 1 1 0 1
            \item Сумма по модулю два $\oplus$\\ 0 1 1 0
            \item Штрих Шеффера $|$\\ 1 1 1 0
            \item Стрелка Пирса $\downarrow$\\ 1 0 0 0
        \end{enumerate}


        \textbf{ДНФ}\\
        \begin{singlespace}
            $\{\vee, \wedge, \overline A\}$
        \end{singlespace}
        \begin{doublespace}
        \end{doublespace}

        \textbf{СДНФ}\\
        \begin{singlespace}
            $f(x_1, x_2) = (x_1 \wedge x_2) \vee (x_1 \wedge \overline x_2)$
        \end{singlespace}
        \begin{doublespace}
        \end{doublespace}


        \textbf{СКНФ}\\
        \begin{singlespace}
            $f(x_1, x_2) = (x_1 \vee x_2) \wedge (x_1 \vee \overline x_2)$
        \end{singlespace}


    \end{minipage}
    \begin{minipage}[t]{0.5\textwidth}
        \centering
        \textbf{Свойства логических операций}
        \begin{enumerate}
            \item Коммутативность\\
            $x \circ y = y \circ x,$\\
            $\circ \in \{\wedge, \vee, \oplus, \downarrow, \leftrightarrow, | \}$

            \item Идемпотентность\\
            $x \circ x = x$\\
            $\circ \in \{\vee, \wedge \}$

            \item Ассоциативность\\
            $(x \circ y) \circ z = x \circ (y \circ z),$\\
            $\circ \in \{\vee, \wedge, \oplus, \leftrightarrow \}$\\

            \item Дистрибутивность\\
            $x \wedge (y \vee z) = (x \wedge y) \vee (x \wedge z)$\\
            $x \vee (y \wedge z) = (x \vee y) \wedge (x \vee z)$\\
            $x \wedge (y \oplus z) = (x \wedge y) \oplus (x \wedge z)$

            \item Законы де Моргана\\
            $\overline{x \wedge y} = \overline x \vee \overline y = x \mid y$\\
            $\overline{x \vee y} = \overline x \wedge \overline y = x \downarrow y$

            \item Законы поглощения\\
            $x \wedge (x \vee y) = x$\\
            $x \vee (x \wedge y) = x$

            \item Другие(1)\\
            \begin{itemize}[topsep=-1cm, leftsep=-1cm, leftmargin=0.1cm]
                \setlength\itemsep{0cm}
                \item $x \oplus x = 0$\\
                \item $x \leftrightarrow x = x \rightarrow x = 1$\\
                \item $x \oplus 0 = x$\\
                \item $x \oplus 1 = x \rightarrow 0 = x \leftrightarrow 0 = x \mid x = x \downarrow x = \overline x$
            \end{itemize}
            \item Другие(2)\\
            \begin{itemize}[topsep=-1cm, leftsep=-1cm, leftmargin=0.1cm]
                \setlength\itemsep{0cm}
                \item $x \oplus y = x \wedge \overline y \vee \overline x \wedge y =$\\
                $(x \vee y) \wedge (\overline x \vee \overline y)$\\

                \item $x \leftrightarrow y  = \overline{x \oplus y} = $\\
                $1 \oplus x \oplus y = x \wedge y \vee \overline x \vee \overline y = $\\
                $= (x \vee \overline y) \wedge (\overline x \vee y)$\\

                \item $x \rightarrow y = \overline x \vee y = x \wedge y \oplus x \oplus 1$

                \item $x \vee y = x \oplus y \oplus x \wedge y$
            \end{itemize}



            $$
        \end{enumerate}
    \end{minipage}

\end{document}